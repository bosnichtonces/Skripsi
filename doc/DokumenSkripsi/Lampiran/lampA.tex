%versi 3 (18-12-2016)
\chapter{Kode Program}
\label{lamp:A}

%terdapat 2 cara untuk memasukkan kode program
% 1. menggunakan perintah \lstinputlisting (kode program ditempatkan di folder yang sama dengan file ini)
% 2. menggunakan environment lstlisting (kode program dituliskan di dalam file ini)
% Perhatikan contoh yang diberikan!!
%
% untuk keduanya, ada parameter yang harus diisi:
% - language: bahasa dari kode program (pilihan: Java, C, C++, PHP, Matlab, C#, HTML, R, Python, SQL, dll)
% - caption: nama file dari kode program yang akan ditampilkan di dokumen akhir
%
% Perhatian: Abaikan warning tentang textasteriskcentered!!
%


\begin{lstlisting}[language=PHP, caption=Controller.php]

<?php

namespace App\Http\Controllers;

use Illuminate\Foundation\Auth\Access\AuthorizesRequests;
use Illuminate\Foundation\Validation\ValidatesRequests;
use Illuminate\Routing\Controller as BaseController;
use Illuminate\Support\Facades\DB;


class Controller extends BaseController
{
	use AuthorizesRequests, ValidatesRequests;
	
	public function home(){
		
		$data_wilayah = DB::table('data_wilayah')->get();
		// dd($data_wilayah->nama_kelurahan);
		return view('homePage',['data_wilayah' => $data_wilayah]);
		
		
	}
}



\end{lstlisting}

\begin{lstlisting}[language=PHP, caption=main.blade.php]
	<!doctype html>
	<html lang="en">
	<head>
	<meta charset="utf-8">
	<meta name="viewport" content="width=device-width, initial-scale=1">
	<title>Pemvisualisasi Area Hijau</title>
	<link href="https://cdn.jsdelivr.net/npm/bootstrap@5.2.3/dist/css/bootstrap.min.css" rel="stylesheet" integrity="sha384-rbsA2VBKQhggwzxH7pPCaAqO46MgnOM80zW1RWuH61DGLwZJEdK2Kadq2F9CUG65" crossorigin="anonymous">
	<script src="https://code.jquery.com/jquery-3.5.1.min.js"></script>
	<style>
	.hidden{
		display:none;
	}
	</style>
	
	<script>
	$(document).ready(function() {
		$(".box").hide();
		$(".box2").hide();
		
		$(".dropdown1").change(function() {
			var optionValue = $(this).val();
			if (optionValue) {
				$(".box").not("." + optionValue).hide();
				$("." + optionValue).show();
			} else {
				$(".box").hide();
			}
		});
		
		$(".dropdown2").change(function() {
			var optionValue = $(this).val();
			if (optionValue) {
				$(".box2").not("." + optionValue).hide();
				$("." + optionValue).show();
			} else {
				$(".box2").hide();
			}
		});
	});
	</script>
	
	<script>
	$(document).ready(function() {
		$('input[name="image-radio"]').change(function() {
			var selectedRadioId = $(this).attr('id');
			
			$('.gambar, .gambar-segmentasi').addClass('hidden');
			
			if (selectedRadioId === 'gambar_kelurahan') {
				$('.gambar').removeClass('hidden');
			} else if (selectedRadioId === 'gambar_kelurahan_segmentasi') {
				$('.gambar-segmentasi').removeClass('hidden');
			}
		});
		
		
		$('input[name="image-radio2"]').change(function() {
			var selectedRadioId = $(this).attr('id');
			
			$('.gambar2, .gambar-segmentasi2').addClass('hidden');
			
			if (selectedRadioId === 'gambar_kelurahan2') {
				$('.gambar2').removeClass('hidden');
			} else if (selectedRadioId === 'gambar_kelurahan_segmentasi2') {
				$('.gambar-segmentasi2').removeClass('hidden');
			}
		});
		
	});
	</script>
	
	
	</head>
	<body>
	
	@include('partials.navbar')
	
	<div class="container mt-3">
	@yield('container')
	</div>
	
	
	<script src="https://cdn.jsdelivr.net/npm/@popperjs/core@2.11.6/dist/umd/popper.min.js" integrity="sha384-oBqDVmMz9ATKxIep9tiCxS/Z9fNfEXiDAYTujMAeBAsjFuCZSmKbSSUnQlmh/jp3" crossorigin="anonymous"></script>
	<script src="https://cdn.jsdelivr.net/npm/bootstrap@5.2.3/dist/js/bootstrap.min.js" integrity="sha384-cuYeSxntonz0PPNlHhBs68uyIAVpIIOZZ5JqeqvYYIcEL727kskC66kF92t6Xl2V" crossorigin="anonymous"></script>
	
	
	
	</body>
	</html>
	
\end{lstlisting}


\begin{lstlisting}[language=PHP, caption=navbar.blade.php]
	<nav class="navbar navbar-expand-lg navbar-dark bg-dark">
	<div class="container">
	<a class="navbar-brand" href="#">Pemvisualisasi</a>
	<button class="navbar-toggler" type="button" data-bs-toggle="collapse" data-bs-target="#navbarNav" aria-controls="navbarNav" aria-expanded="false" aria-label="Toggle navigation">
	<span class="navbar-toggler-icon"></span>
	</button>
	<div class="collapse navbar-collapse" id="navbarNav">
	<ul class="navbar-nav">
	{{-- <li class="nav-item">
			<a class="nav-link {{ ($title == "Home" ? 'active' : '') }}" href="/">Home</a>
			</li> --}}
	</ul>
	</div>
	</div>
	</nav>
	
\end{lstlisting}

\begin{lstlisting}[language=PHP, caption=homePage.php]
	@extends('layout.main')
	
	@section('container')
	
	
	<div class="table-resoponsive">
	<table class="table table-bordered table-fixed">
	<thead>
	<tr>
	<th scope="col" class="col-2">Kelurahan</th>
	<th scope="col" class="col-5">
	<div class="dropdown-center">
	<select class="dropdown1" name="kelurahan" value="Pilih Kelurahan">
	<option value="" selected>Pilih Kelurahan</option>
	@foreach($data_wilayah as $item)
	<option value="{{ str_replace(' ', '_', strtolower($item->nama_kelurahan)) }}">{{ $item->nama_kelurahan }}</option>
	@endforeach
	</select>
	</div>
	</th>
	<th scope="col" class="col-5"> 
	<div class="dropdown-center">
	<select class="dropdown2" name="kelurahan2">
	<option value="" selected>Pilih Kelurahan</option>
	@foreach($data_wilayah as $item)
	<option value="{{ str_replace(' ', '_', strtolower($item->nama_kelurahan)) }}2">{{ $item->nama_kelurahan }}</option>
	@endforeach
	</select>
	</div>
	</th>
	</tr>
	</thead>
	<tbody>
	<tr>
	<th scope="row">Luas Wilayah(km<sup>2</sup>)</th>
	<td style="text-align: center;" class="mx-auto">
	@foreach($data_wilayah as $item)
	<div class="{{ str_replace(' ', '_', strtolower($item->nama_kelurahan)) }} box">{{ $item->luas_kelurahan }}</div>
	@endforeach
	</td>
	
	<td style="text-align: center;" class="mx-auto">
	@foreach($data_wilayah as $item)
	<div class="{{ str_replace(' ', '_', strtolower($item->nama_kelurahan)) }}2 box2">{{ $item->luas_kelurahan }}</div>
	@endforeach
	</td>
	
	</tr>
	
	<tr>
	<th scope="row" class="mx-auto">Luas area hijau(km<sup>2</sup>)</th>
	<td style="text-align: center;">
	@foreach($data_wilayah as $item)
	<div class="{{ str_replace(' ', '_', strtolower($item->nama_kelurahan)) }} box">{{ $item->luas_rth_kelurahan }}</div>
	@endforeach
	</td>
	
	<td style="text-align: center;">
	@foreach($data_wilayah as $item)
	<div class="{{ str_replace(' ', '_', strtolower($item->nama_kelurahan)) }}2 box2">{{ $item->luas_rth_kelurahan }}</div>
	@endforeach
	</td>
	</tr>
	<tr>
	<th scope="row" class="mx-auto">Kebutuhan Luas Area Hijau</th>
	<td style="text-align: center;">
	@foreach($data_wilayah as $item)
	<div class="{{ str_replace(' ', '_', strtolower($item->nama_kelurahan)) }} box"></div>
	@endforeach
	</td>
	
	<td style="text-align: center;" class="mx-auto">
	@foreach($data_wilayah as $item)
	<div class="{{ str_replace(' ', '_', strtolower($item->nama_kelurahan)) }}2 box2"></div>
	@endforeach
	</td>
	</tr>
	<tr>
	<th scope="row" class="mx-auto">Citra Satelit</th>
	<td style="text-align: center;" class="mx-auto">
	@foreach($data_wilayah as $item)
	<div class="{{ str_replace(' ', '_', strtolower($item->nama_kelurahan)) }} box text-center"> 
	<img id="gambar_kelurahan" class="gambar img-fluid" src="{{ asset('image/!!DATA GAMBAR/' . $item->nama_kelurahan . '.png') }}" width="50%" >
	
	<img id="gambar_kelurahan_segmentasi" class="gambar-segmentasi hidden img-fluid" src="{{ asset('image/!!DATA GAMBAR HASIL/' . $item->nama_kelurahan . '.png') }}" alt="Citra Satelit Segmentasi" width="50%" >
	</div>
	@endforeach
	</td>
	
	<td style="text-align: center; " class="mx-auto"> 
	@foreach($data_wilayah as $item)
	<div class="{{ str_replace(' ', '_', strtolower($item->nama_kelurahan)) }}2 box2 text-center"> 
	<img id="gambar_kelurahan2" class="gambar2 img-fluid" src="{{ asset('image/!!DATA GAMBAR/' . $item->nama_kelurahan . '.png') }}" width="50%">
	
	<img id="gambar_kelurahan_segmentasi2" class="gambar-segmentasi2 hidden img-fluid" src="{{ asset('image/!!DATA GAMBAR HASIL/' . $item->nama_kelurahan . '.png') }}" alt="Citra Satelit Segmentasi" width="50%" >
	</div>
	@endforeach
	</td>
	</tr>
	<tr>
	<th scope="row" class="mx-auto">Link Google Maps</th>
	<td style="text-align: center;">
	@foreach($data_wilayah as $item)
	<div class="{{ str_replace(' ', '_', strtolower($item->nama_kelurahan)) }} box"> 
	<a href="{{ $item->link_googlemaps }}" target="_blank">{{ $item->link_googlemaps }}</a>
	</div>
	@endforeach
	</td>
	
	<td style="text-align: center;">
	@foreach($data_wilayah as $item)
	<div class="{{ str_replace(' ', '_', strtolower($item->nama_kelurahan)) }}2 box2"> 
	<a href="{{ $item->link_googlemaps }}" target="_blank">{{ $item->link_googlemaps }}</a>    
	</div>
	@endforeach
	</td>
	</tr>
	<tr>
	<th scope="row" class="mx-auto">Layer</th>
	<td style="text-align: center;">
	<div>
	<input type="radio" name="image-radio" id="gambar_kelurahan"  checked> Citra Satelit
	<input type="radio" name="image-radio" id="gambar_kelurahan_segmentasi" > Segmentasi
	</div>
	</td>
	
	<td style="text-align: center;" class="mx-auto">
	<div>
	<input type="radio" name="image-radio2" id="gambar_kelurahan2"  checked> Citra Satelit
	<input type="radio" name="image-radio2" id="gambar_kelurahan_segmentasi2" > Segmentasi
	</div>
	</td>
	</tr>
	</tbody>
	</table>
	</div>
	
	
	
	@endsection
	

\end{lstlisting}

\begin{lstlisting}[language=PHP, caption=web.php]
	<?php
	
	use Illuminate\Support\Facades\Route;
	use App\Http\Controllers\Controller;
	use App\Http\Controllers\KelurahanController;
	use Illuminate\Support\Facades\Http;
	use Illuminate\Support\Facades\DB;
	
	
	/*
	|--------------------------------------------------------------------------
	| Web Routes
	|--------------------------------------------------------------------------
	|
	| Here is where you can register web routes for your application. These
	| routes are loaded by the RouteServiceProvider and all of them will
	| be assigned to the "web" middleware group. Make something great!
	|
	*/
	
	
	Route::get('/', [Controller::class , 'home' ])->name('home');
	
\end{lstlisting}
%\lstinputlisting[language=PHP, caption=MyCode.java]{./Lampiran/MyCode.java} 

