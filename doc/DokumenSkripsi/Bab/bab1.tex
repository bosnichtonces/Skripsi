%versi 2 (8-10-2016) 
\chapter{Pendahuluan}
\label{chap:intro}
   
\section{Latar Belakang}
\label{sec:label}

%Bagian ini akan diisi dengan apa yang melatarbelakangi pembuatan template skripsi ini.
%Termasuk juga masalah-masalah yang akan dihadapi untuk membuatnya, termasuk kurangnya kemampuan penguasaan \LaTeX{} sehingga template ini dibuat dengan mengandalkan berbagai contoh yang tersebar di dunia maya, yang digabung-gabung menjadi satu jua.
%Bagian lain juga akan dilengkapi, untuk sementara diisi dengan lorem ipsum versi bahasa inggris.

Kebutuhan tempat tinggal bagi masyarakat kota diperlukannya oksigen untuk hidup dan beraktivitas. Agar dapat memenuhi kebutuhan tempat tinggal bagi masyarakat maka pembangunan dan perkembangan pada suatu wilayah harusnya memiliki fungsi lain sebagai Ruang Terbuka Hijau(RTH). RTH merupakan sebuah area yang memanjang atau jalur dan/atau mengelompok,dan penggunaannya yang lebih bersifat terbuka, tempat tumbuh tanaman, baik yang tumbuh secara alamiah maupun yang sengaja ditanam. Salah satu penyumbang oksigen yang besar bagi kota adalah RTH.Tentu saja seharusnya RTH tersedia dalam jumlah yang cukup, terutama pada kota yang memiliki penduduk yang banyak.
%insert image

Pemanfaatan citra satelit merupakan salah satu cara agar dapat mengetahui luas RTH pada suatu kota. Citra Satelit adalah gambaran dari permukaan bumi yang didapatkan lansung dari satelit. Oleh karena itu, citra satelit dapat digunakan dalam pengidentifikasi RTH yang mana terdapatnya banyak pepohonan pada suatu wilayah. Perhitungan juga dapat dilakukan pada citra satelit ,dan hasil dari perhitungan luas RTH pada suatu wilayah diharapkan dapat memberikan dorongan untuk peningkatan dalam penghijauan agar dapat digunakan oleh pemerintah dalam merancang dan meningkatkan penghijauan di berbagai wilayah di Indonesia.
%insert image

Penelitian yang telah dilakukan \VSM, Fritz H. Hutapea SKom, dan Juan A. Kusjadi
%insert reference
menghasilkan data area hijau dari citra satelit pada Kota Bandung yang dibagi menjadi beberapa kelurahan atau kecamatan Kota Bandung. Hasil dari penelitian terdiri dari area hijau untuk 149 kelurahan di 30 kecamatan kota Bandung dan telah selesai dilakukan perhitungan.
%insert image

Pada Skripsi ini, akan dibangun sebuah laman web yang bertujuan untuk pemvisualisasian dari hasil penelitian area hijau Kota Bandung
%insert reference 
.Laman web yang akan dibangun harusnya dapat diakses melalui komputer atau laptop,handphone bersistem android ataupun iOS. Dan dalam pembuatanya akan dibantu dengan menggunakan \emph{Framework} Laravel, sehingga memudahkan pengembang untuk membangun laman web.Dengan adanya laman web pemvisualisasian ini juga akan memudahkan pengguna untuk mengetahui area-area hijau yang ada pada Kota Bandung.

\section{Rumusan Masalah}
\label{sec:rumusan}
Berdasarkan deskripsi dan latar belakang yang sudah dibahas bahwa rumusan masalah yang muncul adalah sebagai berikut:

\begin{itemize}
	\item Bagaimana membuat sebuah laman web interaktif yang dapat membandingkan data dua buah kelurahan atau kecamatan Kota Bandung?
	\item Bagaimana cara pengguna untuk membandingkan atribut-atribut Citra Satelit dari kelurahan atau kecamatan Kota Bandung?
\end{itemize}
%\dtext{6}

\section{Tujuan}
\label{sec:tujuan}
Tujuan dari skripsi ini adalah:
\begin{enumerate}
	\item Membuat sebuah laman web interaktif yang dapat membandingkan dua buah kelurahan.
	\item Pengguna dapat memilih kecamatan atau kelurahan untuk sisi kiri dan kanan, untuk membandingkan atributnya.
\end{enumerate}

%\dtext{7}

\section{Batasan Masalah}
\label{sec:batasan}
%Untuk mempermudah pembuatan template ini, tentu ada hal-hal yang harus dibatasi, misalnya saja bahwa template ini bukan berupa style \LaTeX{} pada umumnya (dengan alasannya karena belum mampu jika diminta membuat seperti itu)

%\dtext{8}
Batasan-batasan masalah untuk penelitian ini adalah sebagi berikut:
\begin{enumerate}
	\item Penilitian dari data yang sudah matang.
\end{enumerate}


\section{Metodologi}
\label{sec:metlit}
Metodologi yang akan digunakan dalam pembuatan skripsi adalah:
\begin{enumerate}
	\item Melakukan survei kepada Fritz H. Hutapea SKom dan Juan A. Kusjadi terkait penenilitiannya
	\item Melakukan pengumpulan data hasil penelitian
	\item Mempelajari ekstraksi data citra satelit yang disimpan pada HDFS
	\item Mempelajari bahasa pemrograman php, html, css dan cara menggunakan \emph{framework} laravel.
	\item Mempelajari kebutuhan laman web.
	\item Melakukan analisis kebutuhan laman web.
	\item Melakukan perancangan antar muka laman web.
	\item Membangun laman web bedasarkan \emph{framework} Laravel.
	\item Melakukan pengujian pada laman web.
	\item Menulis dokumen skripsi.
\end{enumerate}

%\dtext{9}

\section{Sistematika Pembahasan}
\label{sec:sispem}
%Rencananya Bab 2 akan berisi petunjuk penggunaan template dan dasar-dasar \LaTeX.
%Mungkin bab 3,4,5 dapt diisi oleh ketiga jurusan, misalnya peraturan dasar skripsi atau pedoman penulisan, tentu jika berkenan.
%Bab 6 akan diisi dengan kesimpulan, bahwa membuat template ini ternyata sungguh menghabiskan banyak waktu.

Skripsi ini disusun dalam beberapa bab secara sistematis sebagai berikut:
\begin{itemize}
	\item \textbf{Bab 1 Pendahuluan} \\ 
	Berisikan tetang latar belakang, rumusan masalah, tujuan, batasan masalah, metodologi, dan sistematika pembahasan.
	\item \textbf{Bab 2 Landasan Teori} \\ 
	Berisikan tentang dasar-dasar dari teori-teori yang digunakan dalam membangun laman web seperti Ruang terbuka hijau,Citra Satelit,dan \textit{Framework}.
	
\end{itemize}
%\dtext{10}